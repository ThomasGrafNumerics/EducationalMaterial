% e-TeX tools for LaTeX
\usepackage{etoolbox}

% toggle document class
\newtoggle{book}    \settoggle{book}{true}
\newtoggle{exam}    \settoggle{exam}{false}
\newtoggle{article} \settoggle{article}{false}
\newtoggle{beamer}  \settoggle{beamer}{false}

% toggle: document contains citations
\newtoggle{citations}   \settoggle{citations}{true}

% An alternative to babel for XeLaTeX and LuaLaTeX. LOAD BEFORE BIBLATEX
\usepackage{polyglossia}
\setdefaultlanguage[spelling=new, babelshorthands=true]{german}
%\setotherlanguage{latin}

% document contains citations?
%\iftoggle{citations} {
    % Sophisticated Bibliographies in LaTeX
    \usepackage[maxbibnames=99,sorting=none,style=alphabetic,backend=biber]{biblatex}

    \addbibresource{refs.bib}
%2} {}

% Run a document through LaTeX for syntax checking
%\usepackage{syntonly}

% Draw a page-layout diagram
%\usepackage{showframe}

% Advanced font selection in XeLaTeX and LuaLaTeX
\usepackage{fontspec}

% Reimplementation of and extensions to LaTeX verbatim
\usepackage{verbatim}

% is document NOT a beamer presentation?
\iftoggle{beamer} {
    %
} {
    % Flexible and complete interface to document dimensions
    \usepackage[a4paper,bindingoffset=0mm,left=30mm,right=30mm,top=25mm,bottom=25mm,footskip=15mm]{geometry}
}

% Easy access to the Lorem Ipsum dummy text
\usepackage{lipsum}

% Extensive support for hypertext in LaTeX
\usepackage{hyperref}
\hypersetup{
    colorlinks,
    linkcolor={red!40!black},
    citecolor={blue!70!black},
    urlcolor={blue!90!black}
}

\ifboolexpr{not (togl {beamer})}
{
% Control table of contents, figures, etc
\usepackage{tocloft}
}{}

\ifboolexpr{togl {exam} or togl {beamer}}
    {}
    {
        % Extensive control of page headers and footers in LaTeX2ε
        \usepackage{fancyhdr}
    }

% Completely customisable TOCs
\usepackage{etoc}

% Alternative headings for toc/lof/lot
\usepackage{titletoc}

% Conditional commands in LaTeX documents
\usepackage{ifthen}

% String manipulation for (La)TeX
\usepackage{xstring}

% Define commands that appear not to eat spaces
\usepackage{xspace}

% Context sensitive quotation facilities
\usepackage{csquotes}

% AMS mathematical facilities for LaTeX
\usepackage{amsmath}

% TeX fonts from the American Mathematical Society
\usepackage{amssymb}

% AMS-LaTeX commutative diagrams
\usepackage{amscd}

% Typesetting theorems (AMS style)
\usepackage{amsthm}

% Place lines through maths formulae
\usepackage{cancel}

% Access bold symbols in maths mode
\usepackage{bm}

% Macros for manipulating polynomials
\usepackage{polynom}

% Mathematical tools to use with amsmath
\usepackage{mathtools}

\ifboolexpr{not (togl {beamer})}
{
    % Control table of contents, figures, etc
    % Driver-independent color extensions for LaTeX and pdfLaTeX
    \usepackage[svgnames]{xcolor}
    % see 'svgnames' in:
    % https://mirror.koddos.net/CTAN/macros/latex/contrib/xcolor/xcolor.pdf
}{}

% Enhanced support for graphics
\usepackage{graphicx}

% wrapfig – Produces figures which text can flow around
\usepackage{wrapfig}

% Support for sub-captions
\usepackage{subcaption}

% Intermix single and multiple columns
\usepackage{multicol}

% Create PostScript and PDF graphics in TeX
\usepackage{tikz}
% Martin Vogel's Symbols (marvosym) font
\usepackage{marvosym}
% Some symbols created using TikZ
\usepackage[marvosym]{tikzsymbols}
\usetikzlibrary{
  3d,
  arrows,
  arrows.spaced,
  arrows.meta,
  bending,
  babel,
  calc,
  fit,
  patterns,
  patterns.meta,
  plotmarks,
  shapes.geometric,
  shapes.misc,
  shapes.symbols,
  shapes.arrows,
  shapes.callouts,
  shapes.multipart,
  shapes.gates.logic.US,
  shapes.gates.logic.IEC,
  circuits.logic.US,
  circuits.logic.IEC,
  circuits.logic.CDH,
  circuits.ee.IEC,
  datavisualization,
  datavisualization.polar,
  datavisualization.formats.functions,
  er,
  automata,
  backgrounds,
  chains,
  topaths,
  trees,
  petri,
  mindmap,
  matrix,
  calendar,
  folding,
  fadings,
  shadings,
  spy,
  through,
  turtle,
  positioning,
  scopes,
  decorations.fractals,
  decorations.shapes,
  decorations.text,
  decorations.pathmorphing,
  decorations.pathreplacing,
  decorations.footprints,
  decorations.markings,
  shadows,
  lindenmayersystems,
  intersections,
  fixedpointarithmetic,
  fpu,
  svg.path,
  external,
  graphs,
  graphs.standard,
  quotes,
  math,
  angles,
  views,
  animations,
  rdf,
  perspective,
}

% The package provides a PGF/TikZ-based mechanism for drawing linguistic (and other kinds of) trees.
\usepackage{forest}

% Create normal/logarithmic plots in two and three dimensions
\usepackage{pgfplots}
\pgfplotsset{compat=newest}

% Framed environments that can split at page boundaries
\usepackage[framemethod=TikZ]{mdframed}

% Customising captions in floating environments
\usepackage{caption}

% Improved interface for floating objects
\usepackage{float}

% A range of footnote options
\usepackage[hang, flushmargin]{footmisc}

% This package provides user control over the layout of the three basic list environments: enumerate, itemize and description.
\usepackage[shortlabels]{enumitem}
\setlist{nolistsep} % smaller spacing in lists

% A package for producing multiple indexes
\usepackage{imakeidx}

% A generic document command parser
\usepackage{xparse}

% Create PostScript and PDF graphics in TeX
\usepackage{pgf}

% defines \foreach
\usepackage{pgffor}

% This package introduces aliases for counters, that share the same counter register and ‘clear’ list. 
\usepackage{aliascnt}

% Verbatim with URL-sensitive line breaks
\usepackage{url}

% Typeset source code listings using LaTeX
\usepackage{listings}

% float wrapper for algorithms
\usepackage{algorithm}

% layout for algorithmicx
\usepackage[noend]{algpseudocode}

% Extra control of appendices
\usepackage[toc,page]{appendix}

% Change the resetting of counters
\usepackage{chngcntr}

% Highly customised stacking of objects, insets, baseline changes, etc
\usepackage{stackengine}

% Horizontally columned lists
\usepackage{tasks}

% Typeset exercises, problems, etc. and their answers
\usepackage[lastexercise,answerdelayed]{exercise}

% Intelligent cross-referencing. LOAD AFTER \hypersetup
\usepackage[nameinlink,noabbrev]{cleveref}

% left ( and right ) round brackets
\newcommand{\lr}[1]{\left(#1\right)}
% left { and right } curly brackets
\newcommand{\lrc}[1]{\left\{#1\right\}}
% left [ and right ] square brackets
\newcommand{\lrs}[1]{\left[#1\right]}
% set (condition is in math mode): {n\in\N ; n < 10}
\newcommand{\setcm}[2]{\lrc{\;#1 \; ; \; #2 \;}}
% set (condition is in text mode): {n\in\N ; $n$ is even}
\newcommand{\setct}[2]{\lrc{\;#1 \; ; \; \text{#2} \;}}
% set (all in text mode): {$w$ ist binärer String ; $n$ is even}
\newcommand{\settt}[2]{\lrc{\; \text{#1} \; ; \; \text{#2} \;}}
% N (set of natural numbers)
\newcommand{\N}{\mathbb{N}}
% set of units in N = N\{0}
\newcommand{\Nunit}{\N^{\times}}
% Z (set of integers)
\newcommand{\Z}{\mathbb{Z}}
% set of units in Z = Z\{0}
\newcommand{\Zunit}{\Z^{\times}}
% Q
\newcommand{\Q}{\mathbb{Q}}
% set of units in Q = Q\{0}
\newcommand{\Qunit}{\Q^{\times}}
% R
\newcommand{\R}{\mathbb{R}}
% set of units in R = R\{0}
\newcommand{\Runit}{\R^{\times}}
% R >= 0 (non negative real numbers)
\newcommand{\Rnn}{\R_{\geq 0}}
% R > 0 (positive real numbers)
\newcommand{\Rp}{\R_{>0}}
% C
\newcommand{\C}{\mathbb{C}}
% set of units in C = C\{0}
\newcommand{\Cunit}{\C^{\times}}
% K
\newcommand{\K}{\mathbb{K}}
% set of units in K = K\{0}
\newcommand{\Kunit}{\K^{\times}}
% real part of a complex number Re(...)
\newcommand{\real}[1]{\text{Re}\lr{#1}}
% imaginary part of a complex number Im(...)
\newcommand{\imag}[1]{\text{Im}\lr{#1}}
% absolute value
\newcommand{\abs}[1]{\left\lvert#1\right\rvert}
% norm
\newcommand{\norm}[1]{\left\lVert#1\right\rVert}
% closed interval [a,b]
\newcommand{\ic}[2]{\lrs{#1, #2}}
% closed / open interval [a,b)
\newcommand{\ico}[2]{\left[#1, #2\right)}
% open / closed interval (a,b]
\newcommand{\ioc}[2]{\left(#1, #2\right]}
% open interval (a,b)
\newcommand{\io}[2]{\lr{#1, #2}}
% double fraction
\newcommand{\doublefraction}[4]{\cfrac{\frac{#1}{#2}}{\frac{#3}{#4}}}
% degree (of a polynomial)
\newcommand{\degree}[1]{\text{Grad}\lr{#1}}
% defining symbols :=, =:  
\newcommand{\defeql}{\mathrel{\mathop:}=} % defines Symbol :=
\newcommand{\defeqr}{=\mathrel{\mathop:}} % defines Symbol =:
% ceiling operator
\newcommand{\ceil}[1]{\left \lceil #1 \right \rceil}
% floor operator
\newcommand{\floor}[1]{\left \lfloor #1 \right \rfloor}
% quotation marks
\newcommand{\quotes}[1]{``#1''}
% text italics and bold font
\newcommand{\tib}[1]{\textbf{\textit{#1}}}
% implication
\newcommand{\imp}{\Rightarrow}
% cardinality
\newcommand{\anz}[1]{\text{Anz}\lr{#1}}
% arg
\newcommand{\Carg}[1]{\text{arg}\lr{#1}}
% cis
\newcommand{\cis}[1]{\text{cis}\lr{#1}}

% Typeset source code listings using LaTeX
\lstdefinestyle{mystyle}{
    backgroundcolor=\color{BlanchedAlmond!10},   
    commentstyle=\color{Blue},
    keywordstyle=\color{Green},
    numberstyle=\tiny\color{Gray},
    stringstyle=\color{Fuchsia},
    basicstyle=\ttfamily\small,
    breakatwhitespace=false,         
    breaklines=true,                 
    captionpos=b,                    
    keepspaces=true,                 
    numbers=left,                    
    numbersep=5pt,                  
    showspaces=false,                
    showstringspaces=false,
    showtabs=false,                  
    tabsize=4
}
\lstset{style=mystyle}

% Typeset source code listings using LaTeX
\lstdefinestyle{mystyleFootnotesize}{
    backgroundcolor=\color{BlanchedAlmond!10},   
    commentstyle=\color{Blue},
    keywordstyle=\color{Green},
    numberstyle=\tiny\color{Gray},
    stringstyle=\color{Fuchsia},
    basicstyle=\ttfamily\footnotesize,
    breakatwhitespace=false,         
    breaklines=true,                 
    captionpos=b,                    
    keepspaces=true,                 
    numbers=left,                    
    numbersep=5pt,                  
    showspaces=false,                
    showstringspaces=false,
    showtabs=false,                  
    tabsize=4
}

% Typeset source code listings using LaTeX
\lstdefinestyle{mystyleTiny}{
    backgroundcolor=\color{BlanchedAlmond!10},   
    commentstyle=\color{Blue},
    keywordstyle=\color{Green},
    numberstyle=\tiny\color{Gray},
    stringstyle=\color{Fuchsia},
    basicstyle=\ttfamily\tiny,
    breakatwhitespace=false,         
    breaklines=true,                 
    captionpos=b,                    
    keepspaces=true,                 
    numbers=left,                    
    numbersep=5pt,                  
    showspaces=false,                
    showstringspaces=false,
    showtabs=false,                  
    tabsize=4
}

\crefname{lstlisting}{Programm}{Programme}
\Crefname{lstlisting}{Programm}{Programme}
\renewcommand{\lstlistingname}{Listing}

% Python for inline
\newcommand{\pythoninline}[1]{{\lstinline[language=Python]!#1!}}

% C++ for inline
\newcommand{\cppinline}[1]{{\lstinline[language=C++]!#1!}}

\floatname{algorithm}{Algorithmus}

\renewcommand{\algorithmicrequire}{\textbf{Eingabe:}}
\renewcommand{\algorithmicensure}{\textbf{Resultat:}}

% enumerate with roman numbers
\newlist{renum}{enumerate}{1}
\setlist[renum,1]{label=(\roman*)}

% enumerate alphabetically
\newlist{aenum}{enumerate}{1}
\setlist[aenum,1]{label=(\alph*)}

% custom environments
\newcounter{dc} \setcounter{dc}{99999}
\newcommand{\mylabel}{foo} % will get overwritten
\iftoggle{book} {
	\newcounter{cc}[chapter] % cc is reset at start of new chapter
} {
	\newcounter{cc}[section] % cc is reset at start of new section
}
\setcounter{cc}{0}
\iftoggle{book} {
	\renewcommand{\thecc}{\arabic{chapter}.\arabic{cc}}
} {
	\renewcommand{\thecc}{\arabic{section}.\arabic{cc}}
}

% theorem environment
\newaliascnt{ccsatz}{cc}
\aliascntresetthe{ccsatz}
\crefname{ccsatz}{Satz}{Sätze}
\Crefname{ccsatz}{Satz}{Sätze}

\newenvironment{satz}[2][]
    {
        \refstepcounter{ccsatz}
        \ifblank{#1} {
            % if no title is given (#1 blank)
        	\mdfsetup{
        	frametitle={
        	\tikz[baseline=(current bounding box.east),outer sep=0pt]
        	\node[anchor=east,rectangle,fill=Gold!60]
        	{\strut Satz~\thecc};}
        	}
        } {	
            % if a title is given (#1 not blank)
        	\mdfsetup{
        	frametitle={
        	\tikz[baseline=(current bounding box.east),outer sep=0pt]
        	\node[anchor=east,rectangle,fill=Gold!60]
        	{\strut Satz~\thecc~#1};}
        	}
        }
        \mdfsetup{
        	innertopmargin=0pt, linecolor=Gold,
        	linewidth=1pt, topline=true, frametitleaboveskip=\dimexpr-\ht\strutbox\relax
        }
        \ifstrequal{#2}{-} {
        	% #2 is the dummy label "-"
        	\renewcommand{\mylabel}{\arabic{dc}}
        	\refstepcounter{dc}
        } {
        	% #2 is an actual user-defined label
        	\renewcommand{\mylabel}{#2}
        }
        \begin{mdframed}[backgroundcolor=Gold!15]\relax%
        \label{\mylabel}
    } %%
    {
        \end{mdframed}
    }

% definition environment
\newaliascnt{ccdefinition}{cc}
\aliascntresetthe{ccdefinition}
\crefname{ccdefinition}{Definition}{Definitionen}
\Crefname{ccdefinition}{Definition}{Definitionen}

\newenvironment{definition}[2][]
    {
        \refstepcounter{ccdefinition}
        \ifblank{#1} {
            % if no title is given (#1 blank)
        	\mdfsetup{
        	frametitle={
        	\tikz[baseline=(current bounding box.east),outer sep=0pt]
        	\node[anchor=east,rectangle,fill=DarkCyan!40]
        	{\strut Definition~\thecc};}
        	}
        } {	
            % if a title is given (#1 not blank)
        	\mdfsetup{
        	frametitle={
        	\tikz[baseline=(current bounding box.east),outer sep=0pt]
        	\node[anchor=east,rectangle,fill=DarkCyan!40]
        	{\strut Definition~\thecc~#1};}
        	}
        }
        \mdfsetup{
        	innertopmargin=0pt, linecolor=DarkCyan!80,
        	linewidth=1pt, topline=true, frametitleaboveskip=\dimexpr-\ht\strutbox\relax
        }
        \ifstrequal{#2}{-} {
        	% #2 is the dummy label "-"
        	\renewcommand{\mylabel}{\arabic{dc}}
        	\refstepcounter{dc}
        } {
        	% #2 is an actual user-defined label
        	\renewcommand{\mylabel}{#2}
        }
        \begin{mdframed}[backgroundcolor=DarkCyan!8]\relax%
        \label{\mylabel}
    } %%
    {
        \end{mdframed}
    }
    
% vereinbarung environment
\newaliascnt{ccvereinbarung}{cc}
\aliascntresetthe{ccvereinbarung}
\crefname{ccvereinbarung}{Vereinbarung}{Vereinbarungen}
\Crefname{ccvereinbarung}{Vereinbarung}{Vereinbarungen}

\newenvironment{vereinbarung}[2][]
    {
        \refstepcounter{ccvereinbarung}
        \ifblank{#1} {
            % if no title is given (#1 blank)
        	\mdfsetup{
        	frametitle={
        	\tikz[baseline=(current bounding box.east),outer sep=0pt]
        	\node[anchor=east,rectangle,fill=Coral!40]
        	{\strut Vereinbarung~\thecc};}
        	}
        } {	
            % if a title is given (#1 not blank)
        	\mdfsetup{
        	frametitle={
        	\tikz[baseline=(current bounding box.east),outer sep=0pt]
        	\node[anchor=east,rectangle,fill=Coral!40]
        	{\strut Vereinbarung~\thecc~#1};}
        	}
        }
        \mdfsetup{
        	innertopmargin=0pt, linecolor=Coral!80,
        	linewidth=1pt, topline=true, frametitleaboveskip=\dimexpr-\ht\strutbox\relax
        }
        \ifstrequal{#2}{-} {
        	% #2 is the dummy label "-"
        	\renewcommand{\mylabel}{\arabic{dc}}
        	\refstepcounter{dc}
        } {
        	% #2 is an actual user-defined label
        	\renewcommand{\mylabel}{#2}
        }
        \begin{mdframed}[backgroundcolor=Coral!15]\relax%
        \label{\mylabel}
    } %%
    {
        \end{mdframed}
    }

% box environment
\newaliascnt{ccbox}{cc}
\aliascntresetthe{ccbox}
\crefname{ccbox}{Box}{Boxen}
\Crefname{ccbox}{Box}{Boxen}

\newenvironment{myBox}[2][]
    {
        \refstepcounter{ccbox}
        \ifblank{#1} {
            % if no title is given (#1 blank)
        	\mdfsetup{
        	frametitle={
        	\tikz[baseline=(current bounding box.east),outer sep=0pt]
        	\node[anchor=east,rectangle,fill=Indigo!40]
        	{\strut Box~\thecc};}
        	}
        } {	
            % if a title is given (#1 not blank)
        	\mdfsetup{
        	frametitle={
        	\tikz[baseline=(current bounding box.east),outer sep=0pt]
        	\node[anchor=east,rectangle,fill=Indigo!40]
        	{\strut Box~\thecc~#1};}
        	}
        }
        \mdfsetup{
        	innertopmargin=0pt, linecolor=Indigo!40,
        	linewidth=1pt, topline=true, frametitleaboveskip=\dimexpr-\ht\strutbox\relax
        }
        \ifstrequal{#2}{-} {
        	% #2 is the dummy label "-"
        	\renewcommand{\mylabel}{\arabic{dc}}
        	\refstepcounter{dc}
        } {
        	% #2 is an actual user-defined label
        	\renewcommand{\mylabel}{#2}
        }
        \begin{mdframed}[backgroundcolor=Indigo!8]\relax%
        \label{\mylabel}
    } %%
    {
        \end{mdframed}
    }

% exercises and answers
\newaliascnt{ccA}{cc}
\aliascntresetthe{ccA}
\iftoggle{book} {
	\newcounter{ccaufgabe}[chapter]
} {
	\newcounter{ccaufgabe}[section]
}
\setcounter{ccaufgabe}{0}
\iftoggle{book} {
	\renewcommand{\theccaufgabe}{\arabic{chapter}.\arabic{ccaufgabe}}
} {
	\renewcommand{\theccaufgabe}{\arabic{section}.\arabic{ccaufgabe}}
}

% \crefname{ccaufgabe}{Aufgabe}{Aufgaben}
% \Crefname{ccaufgabe}{Aufgabe}{Aufgaben}
\crefname{ccaufgabe}{A}{A}
\Crefname{ccaufgabe}{A}{A}

\newenvironment{aufgabe}[2][]
    {
        \refstepcounter{ccA}
        \refstepcounter{ccaufgabe}
        \ifblank{#1} {
            % if no title is given (#1 blank)
        	\mdfsetup{
            	frametitle={
            	\tikz[baseline=(current bounding box.east),outer sep=0pt]
            	\node[anchor=east,rectangle,fill=DarkSeaGreen]{\strut Aufgabe~\theccA~(A~\theccaufgabe)};},
            	innertopmargin=0pt, linecolor=DarkSeaGreen,
            	linewidth=1pt, topline=true, frametitleaboveskip=\dimexpr-\ht\strutbox\relax
            }
        } {	
            % if a title is given (#1 not blank)
        	\mdfsetup{
            	frametitle={
            	\tikz[baseline=(current bounding box.east),outer sep=0pt]
            	\node[anchor=east,rectangle,fill=DarkSeaGreen]{\strut Aufgabe~\theccA~#1~(A~\theccaufgabe)};},
            	innertopmargin=0pt, linecolor=DarkSeaGreen,
            	linewidth=1pt, topline=true, frametitleaboveskip=\dimexpr-\ht\strutbox\relax
            }
        }
        \mdfsetup{
        	innertopmargin=0pt, linecolor=DarkSeaGreen,
        	linewidth=1pt, topline=true, frametitleaboveskip=\dimexpr-\ht\strutbox\relax
        }
        \ifstrequal{#2}{-} {
        	% #2 is the dummy label "-"
        	\renewcommand{\mylabel}{\arabic{dc}}
        	\refstepcounter{dc}
        } {
        	% #2 is an actual user-defined label
        	\renewcommand{\mylabel}{#2}
        }
        \begin{mdframed}[backgroundcolor=DarkSeaGreen!15]\relax%
        \label{\mylabel}
    } %%
    {
        \end{mdframed}
    }

\newenvironment{antwort}[1]
    {
        \renewcommand{\AnswerHeader}
        {
            \noindent
            \textcolor{DarkSeaGreen!65!black}{\textbf{Lösungsvorschlag}} zu \cref{#1} auf (ab) \cpageref{#1}:\\
            \noindent
        }
        \begin{Answer}
    } %%
    {
        \end{Answer}
    }

% examples
\newaliascnt{ccbeispiele}{cc}
\aliascntresetthe{ccbeispiele}
\crefname{ccbeispiele}{Beispiel}{Beispiele}
\Crefname{ccbeispiele}{Beispiel}{Beispiele}

\makeatletter
\newcommand{\beispiele}[3] {
	\refstepcounter{ccbeispiele}
	\IfEq{#1}{-}
	{
		% no user-defined label given (only "-")
		\renewcommand{\mylabel}{\arabic{dc}}
		\refstepcounter{dc}
	} {
		% actual user-defined label given
		\renewcommand{\mylabel}{#1}
	}
	% do not reference to this if label "-" is given
	\label{\mylabel}\par\smallskip\noindent\textbf{Beispiele~\thecc}\par\noindent #2 \begin{aenum}\item #3\checkNextArgOne}
	\newcommand{\checkNextArgOne}{\@ifnextchar\bgroup{\getNextArgOne}{\end{aenum}}}
	\newcommand{\getNextArgOne}[1]{\item #1\@ifnextchar\bgroup{\getNextArgOne}{~$_\blacktriangle$\smallskip\end{aenum}}
}
\makeatother

% example
\newaliascnt{ccbeispiel}{cc}
\aliascntresetthe{ccbeispiel}
\crefname{ccbeispiel}{Beispiel}{Beispiele}
\Crefname{ccbeispiel}{Beispiel}{Beispiele}

\newcommand{\beispiel}[2] {
	\refstepcounter{ccbeispiel}
	\IfEq{#1}{-}
	{
		% no user-defined label given (only "-")
		\renewcommand{\mylabel}{\arabic{dc}}
		\refstepcounter{dc}
	} {
		% actual user-defined label given
		\renewcommand{\mylabel}{#1}
	}
	% do not reference to this if label "-" is given
	\label{\mylabel}\par\smallskip\noindent\textbf{Beispiel~\thecc}\quad #2 ~$_\blacktriangle$\smallskip
}

% proof
\newcommand{\beweis}[1]{\smallskip\noindent\textbf{Beweis:}\quad #1 ~$_\blacksquare$\smallskip}

% multiple clever-refs
\makeatletter
\newcommand{\getNextArgTwoComma}[1]{,~\hyperref[#1]{\ref*{#1}}\@ifnextchar\bgroup{\getNextArgTwoComma}{\nolinebreak\xspace}}
\newcommand{\cleverPart}[1]{\hyperref[#1]{\namecref{#1}~\labelcref*{#1}}\getNextArgTwoComma}
\newcommand{\cleverParts}[1]{\hyperref[#1]{\namecrefs{#1}~\labelcref*{#1}}\getNextArgTwoComma}
\makeatother

% remarks
\newaliascnt{ccbemerkungen}{cc}
\aliascntresetthe{ccbemerkungen}
\crefname{ccbemerkungen}{Bemerkung}{Bemerkungen}
\Crefname{ccbemerkungen}{Bemerkung}{Bemerkungen}

\makeatletter
\newcommand{\bemerkungen}[3] {
	\refstepcounter{ccbemerkungen}
	\IfEq{#1}{-}
	{
		% no user-defined label given (only "-")
		\renewcommand{\mylabel}{\arabic{dc}}
		\refstepcounter{dc}
	} {
		% actual user-defined label given
		\renewcommand{\mylabel}{#1}
	}
	% do not reference to this if label "-" is given
	\label{\mylabel}\par\smallskip\noindent\textbf{Bemerkungen~\thecc}\par\noindent #2 \begin{aenum}\item #3\checkNextArgOne
}
\makeatother

% remark
\newaliascnt{ccbemerkung}{cc}
\aliascntresetthe{ccbemerkung}
\crefname{ccbemerkung}{Bemerkung}{Bemerkungen}
\Crefname{ccbemerkung}{Bemerkung}{Bemerkungen}

\newcommand{\bemerkung}[2] {
	\refstepcounter{ccbemerkung}
	\IfEq{#1}{-}
	{
		% no user-defined label given (only "-")
		\renewcommand{\mylabel}{\arabic{dc}}
		\refstepcounter{dc}
	} {
		% actual user-defined label given
		\renewcommand{\mylabel}{#1}
	}
	% do not reference to this if label "-" is given
	\label{\mylabel}\par\smallskip\noindent\textbf{Bemerkung~\thecc}\quad #2 ~$_\blacktriangle$\smallskip
}

% axiome
\iftoggle{book} {
	\newcounter{caxiome}[chapter]
} {
	\newcounter{caxiome}[section]
}

\setcounter{caxiome}{0}
\iftoggle{book} {
	\renewcommand{\thecaxiome}{\arabic{chapter}.\roman{caxiome}}
} {
	\renewcommand{\thecaxiome}{\arabic{section}.\roman{caxiome}}
}

\crefname{caxiome}{Axiom}{Axiome}
\Crefname{caxiome}{Axiom}{Axiome}

\makeatletter
\newcommand{\axiome}[3] {
	\refstepcounter{caxiome}
	\IfEq{#1}{-}
	{
		% no user-defined label given (only "-")
		\renewcommand{\mylabel}{\arabic{caxiome}}
		\refstepcounter{caxiome}
	} {
		% actual user-defined label given
		\renewcommand{\mylabel}{#1}
	}
	% do not reference to this if label "-" is given
	\label{\mylabel}\par\smallskip\noindent\textbf{Axiome~\thecaxiom}\par\noindent #2 \begin{aenum}\item #3\checkNextArgOne
}
\makeatother

% gedankenanstoss
\newcommand{\gedankenanstoss}[1]{\bigskip\noindent\textbf{Gedankenanstoss}\quad #1 \par\noindent\hspace{0.9\linewidth}$^\blacktriangle$\bigskip}

% axiom
\newaliascnt{caxiom}{caxiome}
\aliascntresetthe{caxiom}
\crefname{caxiom}{Axiom}{Axiome}
\Crefname{caxiom}{Axiom}{Axiome}

\newcommand{\axiom}[2] {
	\refstepcounter{caxiom}
	\IfEq{#1}{-}
	{
		% no user-defined label given (only "-")
		\renewcommand{\mylabel}{\arabic{caxiom}}
		\refstepcounter{caxiom}
	} {
		% actual user-defined label given
		\renewcommand{\mylabel}{#1}
	}
	% do not reference to this if label "-" is given
	\label{\mylabel}\par\smallskip\noindent\textbf{Axiom~\thecaxiom}\quad #2 ~$_\blacktriangle$\smallskip
}

\iftoggle{book}{
    % BOOK ENVIRONMENT %
    \setcounter{secnumdepth}{5}
    \setcounter{tocdepth}{5}
    % roman chapter numbering instead of normal numbering
    %\renewcommand{\thechapter}{\Roman{chapter}}
    % start chapter numbering at 0
    \setcounter{chapter}{-1}
    % slightly changing chapter marking in header
    \renewcommand{\chaptermark}[1]{\markboth{\textnormal{\thechapter}\ \textnormal{#1}}{}}
    % slightly changing section marking in header
    \renewcommand{\sectionmark}[1]{\markright{\textnormal{\thesection}\ \textnormal{#1}}{}}
    % remove chapter counter in front of section number
    %\counterwithout{section}{chapter}
    % no page number on first page of new chapter
    %\patchcmd{\chapter}{plain}{empty}{}{}
    % resetting the section counter with the start of a new chapter
    %\counterwithin{section}{chapter}
    % do not show chapter number in section numbering
    %\renewcommand*\thesection{\arabic{section}}
    % making header of toc non italics (\upshape) and not all capitals
    \addto\captionsgerman{\renewcommand{\contentsname}{\upshape{I\MakeLowercase{nhaltsverzeichnis}}}}
    % chapter quote at beginning of chapter
    \makeatletter
    \newenvironment{chapquote}[2][2em]
      {\setlength{\@tempdima}{#1}%
       \def\chapquote@author{#2}%
       \parshape 1 \@tempdima \dimexpr\textwidth-2\@tempdima\relax%
       \itshape}
      {\par\normalfont\hfill--\ \chapquote@author\hspace*{\@tempdima}\vspace{1.0cm}\ignorespacesafterend\par\noindent\aftergroup\@doendeq}
    \makeatother
}{}

\iftoggle{exam}{
    % EXAM ENVIRONMENT %
    \pointpoints{Punkt}{Punkte}
    \bonuspointpoints{Bonuspunkt}{Bonuspunkte}
    \renewcommand{\solutiontitle}{\noindent\textbf{Lösung:}\enspace}
    \hqword{Aufgabe:}
    \hpgword{Seite:}
    \hpword{Punkte:}
    \hsword{erhalten:}
    \htword{Total} 
    \bhqword{Aufgabe:}
    \bhpgword{Seite:}
    \bhpword{Bonuspunkte:}
    \bhsword{erhalten:}
    \bhtword{Total} 
    \chqword{Aufgabe:}
    \chpgword{Seite:}
    \chpword{Punkte:}
    \chbpword{Bonuspunkte:}
    \chsword{erhalten:}
    \chtword{Total} 
    %\checkedchar{\CheckedBox}
    \totalformat{Aufgabe \thequestion\ total: \totalpoints\ Punkte}
    \runningheadrule
    %\firstpageheader{\mycourse}{\myclass}{\mydate}
    \runningheader{\mycourse}{\myclass}{\mydate}
    \firstpagefooter{}{\thepage\,/\,\numpages}{}
    \runningfooter{}{\thepage\,/\,\numpages}{}
    
    \CorrectChoiceEmphasis{\normalfont}
    
    %\renewcommand{\questionlabel}{\bfseries Aufgabe~\thequestion.}
}{}




\tikzset{block/.style={
        font=\sffamily,
        draw=black,
        thin,
        fill=pink!50,
        rectangle split,
        rectangle split horizontal,
        rectangle split parts=#1,
        outer sep=0pt},
        %
        gblock/.style={
            block,
            rectangle split parts=#1,
            fill=green!30}
        }

% to highlight matrix entries with tikz
\tikzset{%
  highlight/.style={rectangle,rounded corners,fill=cyan!15,draw=none,
    fill opacity=0.4,thick,inner sep=0pt}
}
\newcommand{\tikzmark}[2]{\tikz[overlay,remember picture,
  baseline=(#1.base)] \node (#1) {#2};}
%
\newcommand{\Highlight}[1][submatrix]{%
    \tikz[overlay,remember picture]{
    \node[highlight,fit=(left.north west) (right.south east)] (#1) {};}
}

\usepackage{drawmatrix}

%\usepackage{mathpazo}
\newcounter{row}
\newcounter{col}

\newcommand\setrow[9]{
  \setcounter{col}{1}
  \foreach \n in {#1, #2, #3, #4, #5, #6, #7, #8, #9} {
    \edef\x{\value{col} - 0.5}
    \edef\y{9.5 - \value{row}}
    \node[anchor=center] at (\x, \y) {\n};
    \stepcounter{col}
  }
  \stepcounter{row}
}

\newcommand\setminirow[4]{
  \setcounter{col}{1}
  \foreach \n in {#1, #2, #3, #4} {
    \edef\x{\value{col} - 0.5}
    \edef\y{4.5 - \value{row}}
    \node[anchor=center] at (\x, \y) {\n};
    \stepcounter{col}
  }
  \stepcounter{row}
}

\usepackage{twemojis}
\usepackage{ifsym}

\newcommand{\romanNumeral}[1]{%
  \textup{\uppercase\expandafter{\romannumeral#1}}%
}

\usepackage{physics}
\usepackage{tikz-3dplot}
\usepackage[outline]{contour} % glow around text

\usepackage[nottoc]{tocbibind}