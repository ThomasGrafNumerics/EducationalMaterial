\chapter{Sortieren (*)}\label{ch:Kapitel05}

Eines der häufigsten algorithmischen Probleme ist das \textit{Sortieren}.
\begin{definition}[Sortierproblem]{definition:Sortierproblem}
    Sei $S$ eine Menge, deren Elemente durch die Relation $\leq$ verglichen werden können ($\leq$ heisst dann \textit{totale Ordnung} auf $S$). Das \tib{Sortierproblem}\index{Sortierproblem} lautet dann:
    \begin{description}
        \item[Eingabe:] Eine endliche Folge von $n\in\Nunit$ Elementen $a_1,a_2,\ldots,a_n$ aus $S$.
        \item[Ausgabe:] Eine Umordnung $a_1', a_2', \ldots, a_n'$ der Eingabe, sodass $a_1'\leq a_2' \leq \ldots \leq a_n'$ (die Elemente der Eingabe sind nun aufsteigend sortiert).
    \end{description}
\end{definition}
\cref{fig:bsp1,fig:bsp2} zeigen zwei einfache Beispiele von Eingaben und entsprechenden Ausgaben des Sortierproblems.

\begin{figure}[H]
    \def\lvld{1.2}                  % Choose level distance
    \pgfmathsetmacro\shft{-6*\lvld} % Calculate the yshift for the green tree
    \centering
\begin{tikzpicture}[level distance=\lvld cm,
                        level 1/.style={sibling distance=4cm},
                        level 2/.style={sibling distance=2cm},
                        level 3/.style={sibling distance=1cm},
                        edgedown/.style={edge from parent/.style={draw=red,thick,-latex}},
                        edgeup/.style={edge from parent/.style={draw=green!50!black,thick,latex-}}
                        ]
        \node[block=7,label={Eingabe}] (A) {3 \nodepart{two} 2 \nodepart{three} 0 \nodepart{four} 3 \nodepart{five} 9 \nodepart{six}8 \nodepart{seven}1}
            [grow=down,edgedown];
        \node[block=7,below=1cm of A,label={Ausgabe}] (B) {0 \nodepart{two} 1 \nodepart{three} 2 \nodepart{four} 3 \nodepart{five} 3 \nodepart{six}8 \nodepart{seven}9}
            [grow=down,edgedown];
\end{tikzpicture}
    \caption{Eingabe und Ausgabe eines kleinen Sortierproblems mit ganzen Zahlen}
    \label{fig:bsp1}
\end{figure}

\begin{figure}[H]
    \def\lvld{1.2}                  % Choose level distance
    \pgfmathsetmacro\shft{-6*\lvld} % Calculate the yshift for the green tree
    \centering
\begin{tikzpicture}[level distance=\lvld cm,
                        level 1/.style={sibling distance=4cm},
                        level 2/.style={sibling distance=2cm},
                        level 3/.style={sibling distance=1cm},
                        edgedown/.style={edge from parent/.style={draw=red,thick,-latex}},
                        edgeup/.style={edge from parent/.style={draw=green!50!black,thick,latex-}}
                        ]
        \node[block=5,label={Eingabe}] (A) {$r$ \nodepart{two} $e$ \nodepart{three} $a$ \nodepart{four} $o$ \nodepart{five} $n$}
            [grow=down,edgedown];
        \node[block=5,below=1cm of A,label={Ausgabe}] (B) {$a$ \nodepart{two} $e$ \nodepart{three} $n$ \nodepart{four} $o$ \nodepart{five} $r$}
            [grow=down,edgedown];
\end{tikzpicture}
    \caption{Eingabe und Ausgabe eines kleinen Sortierproblems mit Kleinbuchstaben des lateinischen Alphabets. Dabei wurde die übliche alphabetische Ordnung $a\leq b\leq \ldots \leq z$ verwendet.}
    \label{fig:bsp2}
\end{figure}

\section{Sortieren mit dem \textit{merge sort} Algorithmus}
Natürlich ist das Bedürfnis nach schnellen Algorithmen besonders gross, falls diese Probleme lösen, die zeitkritisch sind und häufig auftreten. Mittlerweile sind viele Sortieralgorithmen bekannt. Ein berühmter Vertreter dieser Algorithmen ist \textit{merge sort}. Der Algorithmus \textit{merge sort} arbeitet rekursiv. Dabei macht er insbesondere von der Tatsache Gebrauch, dass sich zwei sortierte Listen recht effizient zu einer ebenfalls sortierten Liste zusammenfügen lassen. Es seien also zwei jeweils \textbf{bereits sortierte} Listen \pythoninline{L} und \pythoninline{R} gegeben. Diese Listen besitzen die Längen (Anzahl von Elementen) \pythoninline{len(L)} und \pythoninline{len(R)}. Es ist nun nicht besonders aufwändig, die beiden Listen \pythoninline{L} und \pythoninline{R} zu einer neuen \textbf{sortierten} Liste \pythoninline{M} der Länge \pythoninline{len(L) + len(R)} zu vereinen (englisch: \textit{to merge}).

\begin{aufgabe}{aufgabe:0501}
Angenommen wir haben zwei jeweils \textbf{bereits sortierte} Listen \pythoninline{L} und \pythoninline{R} gegeben. Schreiben Sie eine Python-Funktion \pythoninline{merge(L,R)}, welche \pythoninline{L} und \pythoninline{R} zu einer einzigen sortierten Liste vereint. Ihre Funktion soll unbedingt Gebrauch von der Tatsache machen, dass \pythoninline{L} und \pythoninline{R} bereits sortiert sind.
\begin{lstlisting}[language=Python,caption=Testfälle für die Funktion \pythoninline{merge},numbers=none]
Testfall 0
Eingabe: L = [8, 12, 17], R = []
Ausgabe: [8, 12, 17]

Testfall 1
Eingabe: L = [2, 5, 13], R = [1, 2, 3, 7, 15, 20]
Ausgabe: [1, 2, 2, 3, 5, 7, 13, 15, 20]

Testfall 2
Eingabe: L = [1,10,100,1000], R = [0,5,50]
Ausgabe: [0, 1, 5, 10, 50, 100, 1000]
\end{lstlisting}
Ihre Funktion braucht nicht zu überprüfen, ob \pythoninline{L} und \pythoninline{R} tatsächlich sortiert sind.
\end{aufgabe}

\begin{aufgabe}{aufgabe:0502}
Analysieren Sie den Algorithmus, welchen Sie in \cref{aufgabe:0501} geschrieben haben. Welche Laufzeit hat dieser Algorithmus in Abhängigkeit von \pythoninline{n = len(L) + len(R)}?
\end{aufgabe}
\noindent
Der \textit{merge sort} Algorithmus teilt eine zu sortierende Liste rekursiv so lange in zwei Teile, bis nur noch Listen der Länge 1 vorliegen. Offensichtlich ist jede Liste der Länge 1 bereits sortiert. Schliesslich werden bereits sortierte Listen mit Hilfe des \textit{merge} Algorithmus zu der gesuchten sortierten Liste zusammengefügt. Mit Hilfe der bereits in \cref{aufgabe:0501} erstellten Routine \textit{merge} lässt sich der berühmte \textit{merge sort} Algorithmus in nur wenigen Zeilen in Python beschreiben. Eine mögliche Implementation ist in \cref{listing:mergesort} gegeben.
\begin{lstlisting}[language=Python,caption=Implementation der Funktion \pythoninline{merge_sort},label=listing:mergesort]
def merge_sort(A):
    # sortiert die Liste A
    if len(A) == 1: # Rekursionsanfang
        return A # Listen der Länge 1 sind schon sortiert.

    # teile A in linke Hälfte und rechte Teile
    if len(A) % 2 == 0: # falls len(A) gerade ist
        middle = len(A) // 2
    else: # falls len(A) ungerade ist
        middle = (len(A) // 2) + 1
    L = merge_sort(A[:middle])
    R = merge_sort(A[middle:])
    return merge(L,R)
\end{lstlisting} 
Den \textit{merge sort} Algorithmus zu verstehen ist nicht einfach! Erfahrungsgemäss ist es sehr hilfreich, die einzelnen Schritte dieses Sortieralgorithmus für ein kleines Beispiel durchzugehen. Dazu haben wir die Schritte von \textit{merge sort} (\verb|ms|) beim Sortieren der Liste \pythoninline{[3,2,1]} in \cref{fig:mergesort} schematisch dargestellt. Mit \verb|m| wird natürlich der Algorithmus \textit{merge} bezeichnet. Die schematische Darstellung ist analog zu den Ausführungen in \cref{technisch} zu verstehen.
\begin{figure}[H]
    \centering
    \begin{tikzpicture}[scale=0.7, every node/.style={scale=0.7}, 
        ->, >=stealth', shorten >=1pt, auto, node distance=3.5cm, semithick]
        \tikzstyle{every state}=[fill=Green!10,draw=black,text=black]
        
        \node[state] (ms321) {\verb|ms([3,2,1])|};
        \node[state] (ms32) [below left=2.5cm and 1cm of ms321] {\verb|ms([3,2])|};
        \node[state] (ms3) [below left=1.5cm and 1cm of ms32] {\verb|ms([3])|};
        \node[state] (ms2) [below=1.5cm of ms32] {\verb|ms([2])|};
        \node[state,fill=Gold!15] (m32) [below right=1.5cm and 1cm of ms32] {\verb|m([3],[2])|};
        \node[state] (ms1) [below right=2.5cm and 1cm of ms321] {\verb|ms([1])|};
        \node[state,fill=Gold!15] (m231) [right=2.5cm of ms1] {\verb|m([2,3],[1])|};
        \node[above=0.2cm of ms321] {\textcolor{Blue}{$0$ (anfänglicher Aufruf)}};
    
        \path   (ms321) edge [bend left=15, sloped, midway, above, Blue] node {$1$} (ms32)
                (ms32) edge [bend left=15, sloped, midway, above, Blue] node {$2$} (ms3)
                (ms3) edge [bend left=15, sloped, midway, above, Red] node {$3$} (ms32)
                (ms32) edge [bend left=15, sloped, midway, right, Blue, rotate=-90] node {$4$} (ms2)
                (ms2) edge [bend left=15, sloped, midway, left, Red, rotate=-90] node {$5$} (ms32)
                (ms32) edge [bend left=15, sloped, midway, above, Blue] node {$6$} (m32)
                (m32) edge [bend left=15, sloped, midway, above, Red] node {$7$} (ms32)
                (ms32) edge [bend left, sloped, midway, above, Red] node {$8$} (ms321)
    
                (ms321) edge [bend left=15, sloped, midway, above, Blue] node {$9$} (ms1)
                (ms1) edge [bend left=15, sloped, midway, above, Red] node {$10$} (ms321)
                (ms321) edge [bend left=10, sloped, midway, above, Blue] node {$11$} (m231)
                (m231) edge [bend left=10, sloped, midway, above, Red] node {$12$} (ms321)
        ;
    \end{tikzpicture}
    \caption{schematische Darstellung der Arbeitsschritte von \textit{merge sort}}
    \label{fig:mergesort}
\end{figure}

\section{Analyse der Laufzeit von \textit{merge sort}}
Beachten Sie, dass der in \cref{listing:mergesort} gegebene Algorithmus durchaus in der Lage ist, Listen beliebiger Längen $n\in\N$ zu sortieren. Für die nachfolgenden Untersuchungen wollen wir aber annehmen, dass die Problemgrösse $n$ stets eine Zweierpotenz ist, also $n = 2^k$ für ein $k\in\N$. Ohne diese Vereinfachungen werden in den Untersuchungen diverse Auf- und Abrundeoperationen notwendig sein und die Darstellung wird technischer und weniger instruktiv. Zusätzlich möchten wir den trivialen Fall $n = 0$ als Problemgrösse ausschliessen. Die folgenden Betrachtungen sind inspiriert durch die entsprechenden Abschnitte in dem herausragenden Buch \cite{IntroductionAlgo}.

Die Laufzeit des \textit{merge sort} Algorithmus setzt sich aus drei einzelnen Teilen zusammen:
\begin{description}
  \item[Divide:] Dieser Teil berechnet lediglich die Mitte der zu sortierenden Liste. Dazu ist offensichtlich nur eine konstante Zeit $\Theta(1)$ notwendig.
  \item[Conquer:] Beim Aufruf des \textit{merge sort} Algorithmus mit Problemgrösse $n$ werden rekursiv \textcolor{Blue}{zwei} Teilprobleme (derselben Art) mit jeweils halber Grösse $\textcolor{Green}{n/2}$ aufgerufen. Dies trägt $\textcolor{Blue}{2}T(\textcolor{Green}{n/2})$ zur Laufzeit bei.
  \item[Combine:] Wie wir in \cref{aufgabe:0502} bereits festgestellt haben, benötigt \textit{merge sort} die lineare Zeit $\Theta(n)$ für die Vereinigung zweier Listen mit summierter Länge \pythoninline{n = len(L) + len(R)}.
\end{description}
Zusammengefasst ist die Laufzeit $T(n)$ von \textit{merge sort} im schlimmsten Fall also gegeben durch

\begin{align}\label{eq:rek1}
  T(n) = 
  \begin{cases}
    \Theta(1), &\text{falls $n = 1$,} \\
    2T(n/2) + \Theta(n), & \text{falls $n\geq 2$}.
  \end{cases}
\end{align}
\cref{eq:rek1} lässt sich natürlich schreiben als

\begin{align}\label{eq:rek2}
  T(n) = 
  \begin{cases}
    c_0, &\text{falls $n = 1$,} \\
    2T(n/2) + c_1n, & \text{falls $n\geq 2$}.
  \end{cases}
\end{align}
Diesen Ausdruck können wir mit der Definition $c := \max\lrc{c_0,c_1}$ sogar nochmals vereinfachen zu
\begin{align}\label{eq:rek3}
  T(n) = 
  \begin{cases}
    c, &\text{falls $n = 1$,} \\
    2T(n/2) + cn, & \text{falls $n\geq 2$},
  \end{cases}
\end{align}
da wir uns im Moment lediglich für eine obere Schranke für die Laufzeit $T(n)$ interessieren. Beachten Sie, dass \cref{eq:rek3} die Funktion $T$ (Laufzeit) rekursiv definiert. Gemäss \cref{satz:Rekursion} zum Prinzip der rekursiven Definition existiert genau eine Funktion $T$, welche diese Gleichung erfüllt. Man bezeichnet solch eine rekursiv definierende Gleichung auch als \tib{Rekurrenzgleichung}\index{Rekurrenzgleichung}. 

Wir wollen nun die eindeutige Lösung von \cref{eq:rek3} durch intuitive Überlegungen finden. Betrachten Sie \cref{fig:mergesortBaum}.
\begin{itemize}
  \item In der \textcolor{Red}{obersten} \enquote{Etage} fallen die Kosten $cn$ für den \textit{merge} zu einer Liste der Länge $n$ an.
  \item In der \textcolor{Blue}{zweitobersten} \enquote{Etage} fallen die Kosten $cn/2 + cn/2 = cn$ für zwei \textit{merges} zu Listen der Längen $n/2$ an.
  \item Dies geht rekursiv so weiter.
  \item In der untersten \enquote{Etage} wird überall der Rekursionsanfang erreicht und wir haben $n$-mal die Kosten $T(1) = c$, also $cn$.
\end{itemize}
Diese Überlegung zeigt, dass jede \enquote{Etage} genau $cn$ zu den Gesamtkosten von \textit{merge sort} beiträgt. Nun stellt sich lediglich noch die Frage, wie viele \enquote{Etagen} der Baum in \cref{fig:mergesortBaum} hat. Diese Frage ist aber mit der Frage verwandt, wie häufig eine Zweierpotenz $2^k$ halbiert werden muss, bis das Resultat der Division durch 2 identisch zu 1 ist. Dies ist aber genau das, was uns der Logarithmus zur Basis zwei beantwortet. Der Faktor 2 ist
\begin{align*}
  \log_2(n) = \log_2\lr{2^k} = k
\end{align*}
Male in $n$ enthalten. Das ist die Anzahl Teilungen. Der Baum hat somit $\log_2(n) + 1$ viele \enquote{Etagen}. Die Laufzeit von \textit{merge sort} ist also
\begin{align*}
  cn\lr{\log_2(n) + 1}
\end{align*}
und somit $\Theta\lr{n\log(n)}$.

\begin{figure}[H]
    \centering
\begin{tikzpicture}[
    level/.style={sibling distance=40mm/#1},
    text=black,
    >=latex,
    font=\sffamily
    ]
% https://tex.stackexchange.com/questions/241232/creating-dots-lines-brackets-in-tree-with-tikz
\tikzset{
    edge from parent/.style={draw, thick, black},
    no edge from this parent/.style={
        every child/.append style={
        edge from parent/.style={draw=none}}},
    level 3/.style={yshift=5cm},
    level 4/.style={level distance=5mm} 
         }
  \node (z){\textcolor{Red}{$cn$}} 
  child {node (a) {\textcolor{Blue}{$cn/2$}}
    child {node  (b) {$cn/4$}
      child {node (b1) {$\vdots$}[no edge from this parent]
       child {node (b11) {$c$}}
      }
      child {node (b2) {$\vdots$}[no edge from this parent]
       child {node (b12) {$c$}}
      }
    }
    child {node (g) {$cn/4$}
      child {node (g1) {$\vdots$}[no edge from this parent]
       child {node (g11) {$c$}}
      }
      child {node (g2) {$\vdots$}[no edge from this parent]
       child {node (g12) {$c$}}
      }
    }
  }
    child {node (d) {\textcolor{Blue}{$cn/2$}}
      child {node  (e) {$cn/4$}
        child {node (e1) {$\vdots$}[no edge from this parent]
         child {node (e11) {$c$}}
        }
        child {node (e2) {$\vdots$}[no edge from this parent]
         child {node (e12) {$c$}}
        }
      }
      child {node (f) {$cn/4$}
        child {node (f1) {$\vdots$}[no edge from this parent]
         child {node (f11) {$c$}}
        }
        child {node (f2) {$\vdots$}[no edge from this parent]
         child {node (f12) {$c$}
        }
      }
  }
};

\node[left=5 of z]  (ln1) {$cn$}[no edge from this parent]
    child {node (ln2) {$cn$}[no edge from this parent]
        child {node (ln3) {$cn$}[no edge from this parent]
            child {node (ln4) {}[no edge from this parent]
                child {node (ln5) {$cn$}}}}};

\coordinate (cd1) at ($(f12)+(1,0)$);
\coordinate (nb1) at ($(g12)!.5!(e11)$);

\draw[black,thick,<->,] 
    (cd1) --node [fill=white] {$\log(n)$} (cd1|-z.east);

\draw[black,dashed,thick,->]    
    ($(z.west)+(-1em,0)$) -- (ln1);
\draw[black,dashed,thick,->]    
    ($(a.west)+(-1em,0)$) -- (ln2.east);
\draw[black,dashed,thick,->]    
    ($(b.west)+(-1em,0)$) -- (ln3);
\draw[black,dashed,thick,->]    
    ($(b11.west)+(-1em,0)$) -- (ln5);

\draw[Green!70!black,thick,decorate,decoration={brace,amplitude=10pt,mirror}] (b11.south west) --node [below=0.3cm] {$n$} (f12.south east);
\end{tikzpicture}
    \caption{Laufzeitanalyse von \textit{merge sort}}
    \label{fig:mergesortBaum}
\end{figure}

\begin{aufgabe}{aufgabe:0503}
Zeigen Sie mit Hilfe der vollständigen Induktion, dass
\begin{align*}
    T(n) = n\log_2(n)
\end{align*}
die rekursive Gleichung (Rekurrenz)
\[
  T(n) = 
  \begin{cases}
    2, &\text{falls $n = 2$,} \\
    2T(n/2) + n, & \text{falls $n = 2^k$ für ein $k\in\N, k\geq 2$ erfüllt.}
  \end{cases}
\]
\end{aufgabe}

\begin{comment}
\section{\textit{merge sort} ist optimal}
Es lässt sich relativ leicht beweisen, dass \textit{merge sort} in einem gewissen Sinne eine optimaler (was die Laufzeit betrifft) Sortieralgorithmus ist. Dies ist einer der Gründe, warum der \textit{merge sort} Algorithmus in Schulen und Universitäten gerne untersucht wird. Was meinen wir mit \enquote{in einem gewissen Sinne}? Ein Sortieralgorithmus wird ein (reiner) \textit{vergleichsbasierter} Sortieralgorithmus genannt, falls der Algorithmus lediglich Vergleiche der Form $\leq$ verwendet, um Informationen über die Reihenfolge der zu sortierenden Elemente zu gewinnen. Wir wollen beweisen, dass jeder vergleichsbasiertere Sortieralgorithmus mindestens $n\log(n)$ Vergleiche benötigt, um $n$ Elemente zu sortieren. In diesem Abschnitt nehmen wir an, dass alle zu sortierenden Elemente unterschiedlich sind (ohne Beschränkung der Allgemeinheit).

Wir können die Arbeit vergleichsbasierter Sortieralgorithmen durch binäre Entscheidungsbäume illustrieren. 
\end{comment}


\begin{antwort}{aufgabe:0501}
\begin{lstlisting}[language=Python,caption=Implementation der Funktion \pythoninline{merge},label=listing:merge]
def merge(L,R):
    # L und R sind jeweils bereits sortierte Listen.

    # M wird am Ende len(L) + len(R) viele Elemente haben
    # und eine sortierte Liste sein.
    M = [] # M ist zu Beginn eine leere Liste.

    l = 0 # Laufindex innerhalb der Liste L
    r = 0 # Laufindex innerhalb der Liste R

    # solange weder L noch R vollständig durchlaufen wurden
    while (l < len(L)) and (r < len(R)):
        if L[l] <= R[r]:
            M.append(L[l])
            l = l + 1
        else:
            M.append(R[r])
            r = r + 1

    while l < len(L):
        M.append(L[l])
        l = l + 1
    while r < len(R):
        M.append(R[r])
        r = r + 1
    return M
\end{lstlisting}     
\end{antwort}


\begin{antwort}{aufgabe:0502}
Wir beziehen uns hier auf den Algorithmus in der Musterlösung zu \cref{aufgabe:0501}. Der Algorithmus besteht im Wesentlichen in einer Schleife über die summierte Länge
\begin{center}
    \pythoninline{n = len(L) + len(R)}
\end{center}
der beiden gegebenen Listen. Somit ist die Laufzeit gegeben durch $\Theta(n)$.
\end{antwort}


\begin{antwort}{aufgabe:0503}
Sei also $T(n) := n\log_2(n)$. Es ist zu zeigen, dass diese Wahl von $T$ die Rekurrenz in der Aufgabenstellung erfüllt. Wir untersuchen den Fall $T(2) = T\lr{2^1}$ separat.
Wegen $2\log_2(2) = 2$ ist die Behauptung für $T\lr{2^1}$ korrekt. Nun zeigen wir durch Induktion, dass für alle $k\in\N$ mit $k\geq 2$ die Aussage
\begin{align*}
    \mathcal{A}(k) :\iff T\lr{2^k} = 2T\lr{2^{k-1}} + 2^k
\end{align*}
korrekt ist. Mit den Logarithmengesetzen finden wir
\begin{align*}
    T\lr{2^2} = 2^2\log_2\lr{2^2} = 2^2(\log_2(2) + \log_2(2)) = 2\cdot 2\log_2(2) + 2^2 = 2T\lr{2^1} + 2^2
\end{align*}
und damit ist $\mathcal{A}(2)$ gezeigt. Sei $\mathcal{A}(n)$ nun für ein $n\in\N$ mit $n\geq 2$ wahr, dann finden wir
\begin{align*}
    &T\lr{2^{k+1}} = \\
    &2^{k+1}\log_2\lr{2^{k+1}} = \\
    &2^{k+1}\lr{\log_2\lr{2^{k}} + \log_2(2)} = \\
    &2\cdot 2^{k}\log_2\lr{2^{k}} + 2^{k+1} = \\
    &2T\lr{2^{k}} + 2^{k+1}.
\end{align*}
Dies zeigt den Induktionsschritt.
\end{antwort}


\clearpage
\shipoutAnswer