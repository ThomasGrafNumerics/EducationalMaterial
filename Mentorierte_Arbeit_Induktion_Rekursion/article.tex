
%\renewcommand{\lstlistingname}{Listing}
\section{Vorwort}
Diese Instruktionen beziehen sich in erster Linie auf die Dokumentenklassen \verb|article| und \verb|book|. Vieles hier beschriebene lässt sich (falls sinnvoll) auch in den Dokumentenklassen \verb|exam| und \verb|beamer| verwenden. Für unser Skript werden wir die Dokumentenklasse \verb|book| verwenden.

\section{Kompilation}
Kompiliere das Dokument mit \textcolor{FireBrick}{Lua\LaTeX}.

\section{Toggles}
Zu Beginn des Files \verb|main.tex| kann durch entsprechendes Auskommentieren und Kommentieren die gewünschte Dokumentenklasse gewählt werden. Natürlich muss genau eine der Optionen auskommentiert sein. Ebenso muss am Anfang des Files \verb|preamble.tex| genau der entsprechende Toggle auf \verb|true| gesetzt werden. Für dieses Dokument haben wir in \verb|main.tex| die Option \texttt{\textbackslash}\verb|documentclass[11pt]{article}|
auskommentiert und in \verb|preamble.tex| entsprechend den Toggle \verb|article| durch \verb|\settoggle{article}{true}|
auf \verb|true| gesetzt.

Der Toggle \verb|citations| soll genau dann auf \verb|true| gesetzt werden, falls das Dokument mindestens eine \verb|biblatex|-Referenz enthält. Falls das \verb|biblatex|-Paket importiert wird, obwohl das Dokument keine einzige \verb|biblatex|-Referenz beinhaltet, erhält man die unschöne Warnung
\begin{verbatim}
    The file 'output.bcf' does not contain any citations!
\end{verbatim}

\section{Code}
\begin{lstlisting}[language=Python,caption=Schleife,label=listing:Schleife]
import numpy as np

def foo(n):
    # macht irgendwas
    for k in range(n):
        print('Schleife')
\end{lstlisting}
Wir verwendenden das Paket \verb|listings| um Code in unser Dokument einzufügen.
\cref{listing:Schleife} wird durch den Code
\begin{verbatim}
    \begin{lstlisting}[language=Python,caption=Schleife,label=listing:Schleife]
    import numpy as np
    
    def foo(n):
        # macht irgendwas
        for k in range(n):
            print('Schleife')
    \end{lstlisting}
\end{verbatim}
erzeugt und kann durch \verb|\cref{listing:Schleife}| referenziert werden. Es können auch direkt externe Files eingebunden werden:
\begin{verbatim}
    \lstinputlisting[language=Python]{foo.py}
\end{verbatim}




\pythoninline{for (T thing = foo(); auto& x : thing.items()) { /* ... */ } // OK} \\
\cppinline{for (T thing = foo(); auto& x : thing.items()) { /* ... */ } // OK}

\clearpage


\cref{listing:Schleife}


\begin{satz}{-}{ok}
okii
\end{satz}

\bemerkungen{-}{ok}{oki}{afsfafwew}
\beispiele{okss}{afsfsd}{asfda}{fasfasdfas}
\axiome{adfokss}{afsfsd}{asfda}{fasfasdfas}