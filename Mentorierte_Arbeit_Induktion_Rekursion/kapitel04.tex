\chapter{Binäre Strings ohne aufeinanderfolgende Einsen (*)}\label{ch:Kapitel04}
In \cref{listing:binary} haben wir bereits eine Python-Funktion geschrieben, welche rekursiv alle binären Strings einer gegebenen Länge $n$ ausgibt. Es wird sich ein interessanter und überraschender Zusammenhang zur Fibonacci-Folge ergeben, wenn wir nicht alle binären Strings der Länge $n$ ausgeben, sondern nur die binären Strings der Länge $n$, welche nicht das Muster $11$ (Eins-Eins) enthalten. Wir interessieren uns also nur für die binären Strings, welche nicht zwei aufeinanderfolgende Einsen enthalten.
\begin{aufgabe}{aufgabe:0401}
Ändern Sie \cref{listing:binary} dahingehend ab, dass für gegebenes $n\in\N$ genau die binären Strings der Länge $n$ ohne aufeinanderfolgende Einsen ausgegeben werden. Beginnen Sie auch hier wieder mit dem leeren String und bauen Sie die gesuchten Strings rekursiv auf. Geben Sie Ihrer Funktion die Signatur

\pythoninline{print_binary_without_11(n, w = '')}.
\begin{lstlisting}[language=Python,caption=binäre Strings ohne 11 rekursiv ausgeben,numbers=none]
TESTFALL 0
Eingabe: n = 0, Ausgaben:

(Es wird eine leere Zeile (leerer String) ausgegeben.)
TESTFALL 1
Eingabe: n = 1, Ausgaben:
0
1
TESTFALL 2
Eingabe: n = 2, Ausgaben:
00
01
10
TESTFALL 3
Eingabe: n = 3, Ausgaben:
000
001
010
100
101
\end{lstlisting}
\end{aufgabe}

\clearpage


\section{Anzahl der binären Strings ohne 11}
Sei $n$ eine natürliche Zahl. Wir bezeichnen mit $L_n$ die Menge aller binären Strings der Wörter $n$, die nicht den Teilstring $11$ enthalten, und setzen $N(n) = \abs{L_n}$. Wir wollen $N(n)$, also die Anzahl der Elemente in $L_n$, rekursiv bestimmen. Offensichtlich gilt $N(0) = 1$, denn nur der leere String hat die Länge $0$ und dieser enthält nicht den Teilstring $11$. Des Weiteren gilt $N(1) = 2$, da die Strings $0$ und $1$ beide nicht den Teilstring $11$ enthalten.

\begin{aufgabe}{aufgabe:0402}
Vervollständigen Sie \cref{table:Nn}.
    \begin{table}[H]
        \begin{center}
        \begin{tabular}{c|c}
        $n$ & $N(n)$ \\ \hline
        $0$ & $1$ \\
        $1$ & $2$ \\ 
        $2$ & $3$ \\
        $3$ & $5$ \\
        $4$ & \\
        $5$ & \\ 
        \end{tabular}
        \end{center}
        \caption{Tabelle für $N(n)$}
        \label{table:Nn}
    \end{table}
\end{aufgabe}
\noindent
Betrachten wir nun einen binären String $w$ der Länge $n+1$ mit $n\geq 1$. Angenommen $w$ liegt in $L_{n+1}$, dann enthält offensichtlich auch keiner der Teilstrings von $w$ das Muster $11$. Dann können wir $w\in L_{n+1}$ schreiben als
\begin{align*}
    w = xab,
\end{align*}
wobei $x\in L_{n-1}$ und $a,b\in\lrc{0,1}$.

\begin{aufgabe}{aufgabe:0403}
Wir nehmen an, dass die Anzahlen $N(n)$ und $N(n-1)$ für ein $n$ mit $n\geq 1$ bereits bekannt sind. Wir haben soeben begründet, dass wir $w$ in $L_{n+1}$ schreiben können als
\begin{align*}
    w = xab,
\end{align*}
wobei $x\in L_{n-1}$ und $a,b\in\lrc{0,1}$. Unterscheiden Sie nun zwei Fälle $b = 0$ und $b = 1$ für das Symbol $b$. Drücken Sie $N(n+1)$ durch $N(n)$ und $N(n-1)$ aus.
\end{aufgabe}

\section{Explizite und rekursive Darstellungen von Folgen}
Wir betrachten die rekursiv definierte Folge
\begin{align*}
    a_0 &= -1, \\
    a_{n+1} &= a_n + 4.
\end{align*}
Ein Folgenglied mit grösserem Index, wie zum Beispiel $a_{5000}$, zu berechnen, ist recht mühsam. In dieser rekursiven Darstellung müssten für die Berechnung von 
$a_{5000}$ nämlich alle Glieder $a_{1}, a_{2}, \ldots, a_{4999}$ zuerst bestimmt werden. Wie können wir späte Folgenglieder (mit hohen Indizes) effizienter berechnen? Die folgende \cref{aufgabe:0404} wird sich mit dieser Frage beschäftigen.

\begin{aufgabe}{aufgabe:0404}
Finden Sie eine \enquote{Formel}, welche $a_n$ mit lediglich einer Multiplikation und einer Addition berechnet, ohne zuerst die Vorgänger $a_{1}, a_{2}, \ldots, a_{n}$ bestimmen zu müssen. Berechnen Sie mit Hilfe dieser Formel das Folgenglied $a_{5000}$.
\end{aufgabe}
\noindent
Die in \cref{aufgabe:0404} gefundene (nicht rekursive) Darstellung von $a_n$ wird \tib{explizite Darstellung}\index{Folge!explizite Darstellung einer} von $a_n$ genannt.

\begin{aufgabe}{aufgabe:0405}
Beweisen Sie durch vollständige Induktion, dass $a_n = n^2 -2n$ eine explizite Darstellung der rekursiv definierten Folge
\begin{align*}
    a_0 = 0, \quad a_{n} = a_{n-1} + 2n - 3
\end{align*}
für $n\in\Nunit$ ist.
\end{aufgabe}
\noindent
Die rekursive Darstellung \cref{eq:rekkurenz} ist nicht geeignet, um $N(n)$ für grosse Werte von $n$ zu berechnen. Die Fibonacci-Folge wurde bereits im  Jahr 1202 von Leonardo da Pisa verwendet, um das Wachstum einer Kaninchenpopulation zu beschreiben. Dennoch gelang es (höchst wahrscheinlich) erst in der ersten Hälfte des 18. Jahrhunderts, eine explizite Darstellung dieser wichtigen Folge zu finden. Diese Darstellung zu finden ist also alles andere als einfach. Diese explizite Darstellung ist als \textit{Formel von Moivre-Binet} bekannt und besagt
\begin{align}\label{eq:binet}
    F(n) = \frac{\varphi^n - (1-\varphi)^n}{\sqrt{5}},
\end{align}
wobei $\varphi := \frac{1+\sqrt{5}}{2}$.

\cref{eq:binet} enthält die irrationale Zahl $\sqrt{5}$. Ist es nicht erstaunlich, dass $F(n)$ für alle $n\in\N$ eine natürliche Zahl ist? In der anspruchsvollen \cref{aufgabe:0477} haben Sie die Gelegenheit zu beweisen, dass die Formel von Moivre-Binet tatsächlich die Fibonacci-Zahlen berechnet (und somit ausschliesslich natürliche Zahlen generiert).

\begin{aufgabe}{aufgabe:0478}
Berechnen Sie $F(0)$ und $F(1)$ mit Hilfe von \cref{eq:binet}. Überzeugen Sie sich, dass $F(0)$ und $F(1)$ die ersten beiden Fibonacci-Zahlen und somit insbesondere natürliche Zahlen sind.
\end{aufgabe}

\clearpage
\begin{aufgabe}{aufgabe:0477}
(!) Beweisen Sie durch starke Induktion, dass die $n$-te Fibonacci-Zahl $F_n$ durch den Ausdruck $F(n)$ in \cref{eq:binet} gegeben ist. 

\noindent
Hinweis:
\begin{itemize}
    \item Definieren Sie zuerst
    \begin{align*}
        \alpha := 1-\varphi
    \end{align*}
    und beachten Sie, dass
    \begin{align*}
        \alpha = 1-\varphi = 1 - \frac{1+\sqrt{5}}{2} = \frac{2 - 1 - \sqrt{5}}{2} =  \frac{1-\sqrt{5}}{2}.
    \end{align*}
    \item Beweisen und verwenden Sie nun die beiden Gleichungen
    \begin{align*}
        \varphi^2 &= 1+\phi,\\
        \alpha^2 &= 1+\alpha.
    \end{align*}
\end{itemize}
\end{aufgabe}

\bemerkung{bemerkung:moivre}{
Die Formel von Moivre-Binet enthält Terme, welche Computer aufgrund ihrer (nur) endlichen Arithmetik nicht exakt darstellen können. Eine einfache Beobachtung macht die Formel aber auch für die Berechnung in endlicher Arithmetik gut zugänglich. Wir schreiben zuerst
\begin{align*}
    F(n) = \frac{\varphi^n - (1-\varphi)^n}{\sqrt{5}} = \textcolor{blue}{\frac{\varphi^n}{\sqrt{5}}} - \textcolor{Green}{\frac{(1-\varphi)^n}{\sqrt{5}}}.
\end{align*}
Die Fibonacci-Zahl $F_n$ unterscheidet sich von der Zahl $\textcolor{blue}{\frac{\varphi^n}{\sqrt{5}}}$ also lediglich um den Term $\textcolor{Green}{\frac{(1-\varphi)^n}{\sqrt{5}}}$. Doch wie gross ist dieser Term? Für den Exponenten $n=0$ reduziert er sich auf
\begin{align*}
    \abs{\frac{1-\varphi}{\sqrt{5}}}.
\end{align*}
Wegen $\abs{1-\varphi} < 1$ gilt
\begin{align*}
    \abs{\frac{1-\varphi}{\sqrt{5}}} < \frac{1}{\sqrt{5}} < \frac{1}{2}.
\end{align*}
Was ändert sich aber, wenn wir allgemeine Exponenten $n\in\N$ zulassen? Tatsächlich gilt: Je grösser der Exponent ist, desto kleiner ist der Term! Aufgrund der Abschätzung $\abs{1-\varphi} < 1$ wissen wir nämlich, dass die geometrische Folge
\begin{align*}
    \abs{\textcolor{Green}{\frac{(1-\varphi)^n}{\sqrt{5}}}} = \frac{1}{\sqrt{5}}\abs{(1-\varphi)^n} = \frac{1}{\sqrt{5}}\abs{1-\varphi}^n
\end{align*}
strikt monoton fallend ist. Damit gilt also für alle natürlichen Exponenten $n\in\N$
\begin{align*}
    \abs{\textcolor{Green}{\frac{(1-\varphi)^n}{\sqrt{5}}}} < \frac{1}{2}. 
\end{align*}
Wir haben somit nachgewiesen, dass
\begin{alignat*}{2}
    \textcolor{blue}{\frac{\varphi^n}{\sqrt{5}}} - \frac{1}{2} &< \underbrace{\textcolor{blue}{\frac{\varphi^n}{\sqrt{5}}} - \textcolor{Green}{\frac{(1-\varphi)^n}{\sqrt{5}}}}_{F_n} &&< \textcolor{blue}{\frac{\varphi^n}{\sqrt{5}}} + \frac{1}{2} \iff \\
    - \frac{1}{2} &< \quad\; F_n - \textcolor{blue}{\frac{\varphi^n}{\sqrt{5}}} &&< \frac{1}{2},
\end{alignat*}
oder anders geschrieben:
\begin{align*}
    \abs{F_n-\textcolor{blue}{\frac{\varphi^n}{\sqrt{5}}}} < \frac{1}{2}.
\end{align*}
Der Abstand von $\textcolor{Blue}{\varphi^n/\sqrt{5}}$ zu der ganzen Zahl $F_n$ ist also kleiner als $1/2$. Es liegt damit keine ganze Zahl so nahe bei $\textcolor{Blue}{\varphi^n/\sqrt{5}}$ wie $F_n$. Die $n$-te Fibonacci-Zahl $F_n$ entspricht somit $\textcolor{Blue}{\varphi^n/\sqrt{5}}$, gerundet auf die nächste ganze Zahl:
\begin{align}\label{eq:moivrecomp}
    F_n = \lrs{\textcolor{blue}{\frac{\varphi^n}{\sqrt{5}}}} \quad\text{für alle } n\in\N.
\end{align}
Damit ist nun auch klar ersichtlich, dass die Fibonacci-Folge exponentiell wächst!
}

\begin{aufgabe}{aufgabe:0406}
Verwenden Sie die Formel von Moivre-Binet, um eine explizite Darstellung von $N(n)$ zu finden und schreiben Sie eine Python-Funktion \pythoninline{def N(n)}, welche $N(n)$ für gegebenes $n\in\N$ berechnet. Verwenden Sie dazu \cref{eq:moivrecomp}.
\end{aufgabe}

\begin{antwort}{aufgabe:0401}
\begin{lstlisting}[language=Python,caption=binäre Strings ohne $11$]
def print_binary_without_11(n, w = ''):
    if n == 0:
        print(w)
        return
    
    if len(w) == 0 or  w[-1] != '1':
        print_binary_without_11(n-1, w + '0')
        print_binary_without_11(n-1, w + '1')
    else:
        print_binary_without_11(n-1, w + '0')
\end{lstlisting}
\end{antwort}

\begin{antwort}{aufgabe:0402}
    \begin{table}[H]
        \begin{center}
        \begin{tabular}{c|c}
        $n$ & $N(n)$ \\ \hline
        $0$ & $1$ \\
        $1$ & $2$ \\ 
        $2$ & $3$ \\
        $3$ & $5$ \\
        $4$ & $8$ \\
        $5$ & $13$ \\ 
        $6$ & $21$ \\
        $7$ & $34$ \\
        $8$ & $55$ \\
        \end{tabular}
        \end{center}
        \caption{ausgefüllte Tabelle für $N(n)$}
        \label{table:Nncompleted}
    \end{table}
\end{antwort}


\begin{antwort}{aufgabe:0403}
    Wir unterscheiden zwei mögliche Fälle für das Symbol $b$.
    \begin{itemize}
        \item Falls $b = 0$, dann hat $w$ die Form $w = xab = xa0$. Da $b = 0$ ist, kann $a$ sowohl $0$ als auch $1$ sein (es gibt keine Einschränkung für $a$). Dann hat also $w$ die Form
        \begin{align*}
            w = \underbrace{xa}_{=:y}0 = y0,
        \end{align*}
        wobei $y$ ein beliebiger String aus $L_n$ sein darf, und $L_n$ enthält $N(n)$ Elemente.
        \item  Falls $b = 1$, dann muss $a = 0$ gelten. Für $x$ ist dann aber ein beliebiger String aus $L_{n-1}$ möglich und $L_{n-1}$ enthält $N(n-1)$ Elemente.
    \end{itemize}
    Da für $b$ genau zwei Fälle möglich sind und diese Fälle keine \enquote{Überlappung} aufweisen, können wir die jeweiligen Anzahlen addieren und erhalten die Gleichung
    \begin{align}\label{eq:rekkurenz}
        N(n+1) = N(n) + N(n-1).
    \end{align}
    Dies sieht aber genauso aus wie die rekursive Definition der Fibonacci-Folge! Aufgrund der bereits gefundenen Anfangsbedingungen $N(0) = 1$ und $N(1) = 2$ muss die gewöhnliche Fibonacci-Folge \enquote{verschoben} werden und wir sehen
    \begin{align*}
        N(n) = F(n+2).
    \end{align*}
\end{antwort}


\begin{antwort}{aufgabe:0404}
Wir bemerken, dass das nachfolgende Folgenglied um 4 grösser ist als sein Vorgänger. Damit ist $a_n$ genau $4n$ grösser als $a_0$ und wir erhalten
\begin{align*}
    a_n = a_0 + 4n = 4n - 1.
\end{align*}
Nun finden wir $a_{5000} = 4\cdot 5000 - 1 = 19999$.
\end{antwort}



\begin{antwort}{aufgabe:0405}
Wegen $a_0 = 0^2 - 2\cdot 0 = 0$ ist der Induktionsanfang gezeigt. Angenommen die Aussage gilt für $n$. Wir zeigen, dass sie auch für $n+1$ gilt, dass also $a_{n+1} = (n+1)^2 - 2(n+1) = n^2 - 1$. In der Tat ist
\begin{align*}
    a_{n+1} = a_n + 2(n+1) - 3 = n^2 - 2n + 2(n+1) - 3 = n^2 - 1.
\end{align*}
\end{antwort}


\begin{antwort}{aufgabe:0478}
Wir berechnen
\begin{align*}
    F(0) &= \frac{\varphi^0-\alpha^0}{\sqrt{5}} = 0 = F_0, \\
    F(1) &= \frac{\varphi^1-\alpha^1}{\sqrt{5}} = \frac{\varphi-(1-\varphi)}{\sqrt{5}} = \frac{2\varphi-1}{\sqrt{5}} = \frac{1+\sqrt{5}-1}{\sqrt{5}} = 1 = F_1.
\end{align*}
\end{antwort}


\begin{antwort}{aufgabe:0477}
Wir beweisen zuerst die Gleichungen aus dem Hinweis:
\begin{align*}
    \varphi^2 &= \lr{\frac{1+\sqrt{5}}{2}}^2 = \frac{6+2\sqrt{5}}{4} = 1 + \frac{1+\sqrt{5}}{2} = 1 + \varphi, \\
    \alpha^2 &= \lr{\frac{1-\sqrt{5}}{2}}^2 = \frac{6-2\sqrt{5}}{4} = 1 + \frac{1-\sqrt{5}}{2} = 1 + \alpha.
\end{align*}
In \cref{aufgabe:0478} haben wir bereits die Korrektheit der Aussage für $n=0$ und $n=1$ gezeigt. Sei nun die Aussage für ein $n\in\Nunit$ wahr.
\begin{align*}
    F_{n+1} & \stackrel{\text{Definition von $F_{n+1}$}}{=} F_n + F_{n-1} = \\
    &\frac{\varphi^{n}-\alpha^{n}}{\sqrt{5}} + \frac{\varphi^{n-1}-\alpha^{n-1}}{\sqrt{5}} = \\
    &\frac{\varphi^{n} + \varphi^{n-1} - \lr{\alpha^{n}+\alpha^{n-1}}}{\sqrt{5}} = \\
    &\frac{\lr{\varphi + 1}\varphi^{n-1} - \lr{\lr{\alpha + 1}\alpha^{n-1}}}{\sqrt{5}} = \\
    &\frac{\varphi^2\varphi^{n-1} - \alpha^2\alpha^{n-1}}{\sqrt{5}} = \\
    &\frac{\varphi^{n+1}-\alpha^{n+1}}{\sqrt{5}} = \frac{\varphi^{n+1}-(1-\varphi)^{n+1}}{\sqrt{5}} = F(n+1)
\end{align*}
\end{antwort}


\begin{antwort}{aufgabe:0406}
Wir haben bereits festgestellt, dass
\begin{align*}
    N(n) = F(n+2) = \frac{\varphi^{n+2}-(1-\varphi)^{n+2}}{\sqrt{5}}
\end{align*}
mit $\varphi := \frac{1+\sqrt{5}}{2}$ gilt. Das entsprechende Programm lautet:
\begin{lstlisting}[language=Python]
import math

def N(n):
    sqrt5 = math.sqrt(5)
    phi = (1 + sqrt5)/2
    return int(round(phi**(n+2)/sqrt5))
\end{lstlisting}
\end{antwort}


\clearpage
\shipoutAnswer